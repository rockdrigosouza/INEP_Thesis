\onehalfspacing %Para um espa�amento de 1,5
\chapter{Cap�tulo 02}\label{Ch2}
  
Os Anais do CONEM 2010 ser�o publicados em CDROM, usando o formato Adobe$^{TM}$ PDF.

Os artigos devem ser rigorosamente formatados de acordo com estas instru��es e este arquivo texto pode ser usado como um template por usu�rios do Microsoft Word$^{TM}$ e, em qualquer caso, como um modelo para os usu�rios de outros softwares processadores de texto.

Os artigos est�o limitados a um m�ximo de 10 p�ginas, incluindo tabelas e figuras. O arquivo final em formato pdf n�o deve exceder 2,5 MB.

A l�ngua oficial do congresso � o Portugu�s, entretanto ser�o aceitos manuscritos em Espanhol ou em Ingl�s. Se o trabalho n�o for escrito em ingl�s, o autor dever� incluir o t�tulo, os nomes dos autores e afilia��es, o resumo e as palavras-chave, traduzidos para o ingl�s, ap�s a lista de refer�ncias, no fim do artigo.

\section{Teste2}
Texto de se��o para teste

\subsection{Teste3}
Texto de subse��o para teste
\begin{citacao}
 Este � um exemplo de cita��o. S� utilize este ambiente se
 a sua cita��o tiver mais de 3 linhas.
\end{citacao}


\subsubsection{Teste4}
Texto de subsubse��o para teste

\paragraph{Teste5}
Texto de subsubsubse��o para teste

\paragraph{Teste6}

aqui segue o barco do par�grafo normal

  

%%%%%%%%%%%%%%%%%%%%%%%%%%%%%%%%%%%%%%%%%%%%%%%%%%%%%%%%%%%%%%%%%%%%%%%%%%%%%%%%%%%%%%%%%%%%%%%%%%
%%%%%%%%%%% END OF FILE
%%%%%%%%%%%%%%%%%%%%%%%%%%%%%%%%%%%%%%%%%%%%%%%%%%%%%%%%%%%%%%%%%%%%%%%%%%%%%%%%%%%%%%%%%%%%%%%%%%

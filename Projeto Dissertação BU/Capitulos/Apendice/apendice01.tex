\chapter{Discretiza��o do controlador PI}\label{tustincontrol}
%\cite{Denizar,IvoBarbi01}

Dada  a fun��o do controlador PI:
\[
	PI(s)=K_{PI}\left(\frac{s+\omega _z}{s}\right)
\]
Pelo m�todo de discretiza��o de Tustin a vari�vel $s$ deve ser substitu�da por:
\[
s=m\frac{z-1}{z+1}
\]
Desta maneira:
\[
	PI(z)=K_{PI}\frac{m\displaystyle\frac{z-1}{z+1}+\omega _z}{m\displaystyle\frac{z-1}{z+1}}
\]
Que pode ser reescrita como:
\[
PI(z)=K_{PI}\frac{mz-m+\omega _zz+\omega _z}{m(z-1)}
\]
Agrupando os coeficientes de termos semelhantes e multiplicando numerador e denominador por $z^{-1}$:
\[
PI(z)=K_{PI}\frac{z(m+\omega _z)+(\omega _z-m)}{m(z-1)}\cdot \frac{z^{-1}}{z^{-1}}
\]
O que resulta em:
\[
PI(z)=K_{PI}\frac{(m+\omega _z)+z^{-1}(\omega _z-m)}{m-mz^{-1}}
\]
Fazendo as devidas manipula��es matem�ticas e sabendo que $z$ representa um atraso unit�rio no tempo:

\[
mu[k]-mz^{-1}u[k]=K_{PI}(e[k](\omega_z+m)+z^{-1}e[k](\omega_z-m))
\]

\[
mu[k]-mu[k-1]=K_{PI}(e[k](\omega_z+m)+e[k-1](\omega_z-m))
\]

\[
mu[k]=mu[k-1]+K_{PI}(e[k](\omega_z+m)+e[k-1](\omega_z-m))
\]

\[
u[k]=u[k-1]+\frac{K_{PI}}{m}(\omega_z+m)e[k]+\frac{K_{PI}}{m}(\omega_z-m)e[k-1]
\]

Definindo as vari�veis $A$ e $B$:

\[
A=\frac{K_{PI}}{m}(\omega_z-m)
\]

\[
B=\frac{K_{PI}}{m}(\omega_z+m)
\]

A equa��o a diferen�as resultante, que ser� implementada no DSC, �:

\[
u[k]=u[k-1]+Be[k]+Ae[k-1]
\]